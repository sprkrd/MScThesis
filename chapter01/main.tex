\documentclass[../root.tex]{subfiles}

\begin{document}

\section{Introduction}

\subsection{Overview}
\label{sec:overview}

Industrial \emph{robots} are programmed to perform highly specialized and
repetitive actions in controlled environments. Therefore, their
autonomy is quite limited and they are restricted to a small set of
problems. We would like that robots that are
not in such controlled scenarios are able solve a larger variety of
problems, and even to react in the presence
of adverse events.
In other words, we
would like robots to reason about its environment and about the
potential effects that their actions may exert on it.
Such capabilities would allow for great flexibility in the following
ways: (1) tasks can be switched without reprogramming; (2) multiple
solutions to the same problem can be found and assessed in terms
of risk and potential gains in execution speed or other criteria; and (3)
contingency mechanisms can be applied, should adverse events hinder
the execution of the task.

We see in the area of \emph{Automatic Planning and Scheduling},
or simply planning, tools that potentially bring us toward these
desirable qualities.
Planning it is a branch inside Artificial
Intelligence that aims at solving problems that are defined
declaratively, with as little control (procedural) knowledge
as possible.
That is, a
planning algorithm
operates with a model that encodes the dynamics of the domain and, for
each problem, outputs
an action sequence
or a policy that solves it, has a high success probability or has a high
expected reward.

Since Automatic Planning deals with optimizing an agent's
behavior to operate in a certain environment, it is possible to see some
similarities between planning and Reinforcement Learning. The main difference
is that, in their most basic form, a Reinforcement Learning agent is
model-free and dependent on its learning capabilities to decide the
suitable action for a certain state, while a planner takes advantage of a
model of the domain and employs informed search or other heuristic methods to
find a plan (although definitively there
exist works that allow partial domains and that incorporate learning
capabilities to complete these them~\cite{martinez2017relational,martinez2015vmin}).

It is worth saying that planning is not a discipline that is particular to
robotics and, therefore, there exist multiple benchmarks and work in the
field that is general enough to be of use in many areas. One of the
most successful conferences that spreads knowledge in the field of planning is
ICAPS (International Conference in Automatic Planning and Scheduling)
\footnote{\url{http://www.icaps-conference.org/}}. The conference is
held every year and hosts a competition that seeks to push the boundaries
of the state of the art planning systems.

We can think of several applications that are potential targets
of these advances: assistance of old or impaired people for household chores
or treatment~\cite{canal2018adapting,andriella2018deciding}; automatic system
maintenance in difficult-to-access environments, like
subaquatic facilities~\cite{palomeras2016toward,ong2010planning}; and the one
in which we focus on: to \emph{automatically disassembly} electronic devices,
retrieving their most valuable components. As we will see in the following
section, the recycling application is very exciting both from a scientific
and environmental one.

We will resort to the MDP (\emph{Markov Decision Process})
formalism to express our problem, capturing the uncertainty over
the state transitions.
One of the main limitations of MDPs for
real-life problems is that they assume full observability of the state variables.
In practice, this is not true for our application. However, dropping this
assumption results in POMDPs
(\emph{Partially Observable MDPs}), whose resolution is computationally costly
and scales badly with the size of the state space. Moreover, POMDPs are
still difficult to apply when the agent does not know all
the variables of the state (which may happen when the it is not aware of all
the relevant objects of the scene).

We update and solve a new MDP each
time the specification of the state changes (i.e. when new objects are discovered
after performing an action). Even though solving MDPs is much less intensive than
solving POMDPs, conventional algorithms such as \emph{Value Iteration} and
\emph{Policy Iteration} still suffer from the curse of dimensionality when there
are many state variables. Then, computing a full policy becomes infeasible.
We explore two complementary strategies for dealing with this: (1)
determinization algorithms together with classical planners; and (2)
decompositon of the main goal into sub-goals, taking advantage of the
topological precedences imposed by some circumstances (e.g. partial
occlusion).

\subsection{Domain of application}

The research presented in this document has been conducted aiming at
developing techniques that can be effectively employed to \emph{recycle}
contraptions such as hard drives, hair trimmers, remote controls or
electronic toys. Fig.~\ref{fig:examples-of-devices} shows some
examples of the devices that we would like to disassemble.

\begin{figure}[tbhp]
	\centering
	\begin{subfigure}[b]{0.31\columnwidth}
		\includegraphics[width=\textwidth]{hddbottom}
		\caption{}
	\end{subfigure}
	~
	\begin{subfigure}[b]{0.31\columnwidth}
		\includegraphics[width=\textwidth]{gsmamp}
		\caption{}
	\end{subfigure}
	~
	\begin{subfigure}[b]{0.31\columnwidth}
		\includegraphics[width=\textwidth]{handies}
		\caption{}
	\end{subfigure}
	\caption{
		(a) Bottom view of a hard drive being disassembled.
		(b) GSM amplifier being disassembled.
		(c) Several handie-talkie models being displayed.
	}
	\label{fig:examples-of-devices}
\end{figure}

In our view, this application poses very attractive challenges
from a scientific point of view. In real-life, deterministic environments
(i.e. fully observable state variables and non-random action outcomes)
are more an exception than a rule, and this becomes evident in disassembly
scenarios. On the one hand, the whole structure of the device cannot be
perceived at once because of occlusions and perception noise, as
Fig.~\ref{fig:examples-of-devices} illustrates. Therefore, progression
in the disassembly task is required to discover hidden components and
geometrical relations. On the other hand, actions might not produce
always the same outcome, due to positioning errors, noise in the movement
of the robots and exogenous factors. For instance, levering the PCB
of a hard drive in order to extract it from the case might not success
entirely, leaving the PCB hovering over one of the edges of the case.
In such case, one may need to consider an additional action like levering
from a different point or holding the case upside down to let the loosen
component fall.

\begin{figure}[tbhp]
	\centering
	\begin{subfigure}[b]{0.45\columnwidth}
		\includegraphics[width=\textwidth]{hddtopcover}
		\caption{}
	\end{subfigure}
	~
	\begin{subfigure}[b]{0.45\columnwidth}
		\includegraphics[width=\textwidth]{hddtopwocover}
		\caption{}
	\end{subfigure}
	\caption{
		(a) Top view of a hard drive with the lid.
		(b) View from the same perspective without the lid. The
		previously occluded inner components are now visible.
	}
	\label{fig:example-of-occlusion}
\end{figure}

We cannot forget either the social and environmental impact of the
proposed application. The current recycling industry is dominated
by the \emph{crush and separate} paradigm. Trash is crushed into
very small bits that are then filtered and classified using
physical properties such as density or inductance. However,
the uniformity of the separated debris cannot be guaranteed.
Even more importantly, devices often contain components or
substances that are hazardous for the environment. In such cases,
it is require that a human manually removes the source of danger.
This has associated health risks for the operator. Moreover,
it is inefficient and costly, so a great amount of trash is
simply incinerated, posing a serious thread for the ecosystems.
Another drawback of this method is that precious or reusable
components (e.g. PCBs, capacitors, magnets...) are destroyed,
when they can be used right away or after some refurnishing
in the manufacturing of new products.

In light of these arguments, a robotic recycling system is
appealing from a sustainability point of view.
To solve the previously exposed drawbacks, we would like
the following features: (1) the robot should
ideally adapt to different electromechanical contraptions
(including different brands/models of the same device);
(2) the robot should correctly identify the tasks that it
has to complete to perform a successful disassembly; and (3)
it should be able to work even with damaged devices.

\section{Involvement in H2020 Imagine}

This research has been conducted as part of the European H2020 project
\emph{Imagine: Robots Understanding Their Actions by Imagining Their Effects},
or just \emph{Imagine}\footnote{\url{https://imagine-h2020.eu/}}. The scientific
objective of \emph{Imagine} is to make robots aware of their environment
and, in general, to achieve the autonomy objectives described in
Section~\ref{sec:overview}. The project focuses on the recycling
application introduced in the former section.

One of the main novel ideas of the project is to use a physics simulator and an
association engine
to simulate the different actions that the robot may exert upon the environment.
This is useful to ``imagine'' plans before executing them and identifying
the useful actions. Another key aspect of the project is the concept
of ADES, or \emph{Action Description}, for storing the symbolic description
of the actions and the DMPs (\emph{Dynamic Movement Primitive}) associated to
each one. Other aspects such as the perception of the scene and the
design and construction of a multi-funtional gripper that can deal
successfully with a wide range of device fall inside the scope of \emph{Imagine}.
The work presented in this thesis is the result of our research efforts toward
designing and implementing a decision making subsystem for \emph{Imagine}. This
subsystem must consider all the available information from the scene to select
a suitable action and 

\begin{wrapfigure}{L}{0.4\columnwidth}
	\centering
	\includegraphics[width=0.95\linewidth]{vrepimage}
	\caption{Execution of a levering action in the V-REP simulator.}
	\label{fig:vrepimage}
\end{wrapfigure}

There are seven European research centers participating in \emph{Imagine},
each working on a different subsystem.
The planning subsystem is developed by the IRI-CSIC 
(\emph{Institut de Rob\`otica i Inform\`atica
Industrial}\footnote{\url{http://www.iri.upc.edu/}}-\emph{Consejo Superior
de Investigaciones
Cient\'ificas}\footnote{\url{http://www.csic.es/}}) partner, which also hosts
this thesis.

As of today, we have been participating actively on \emph{Imagine} for more
than a year.
In April the project was evaluated by a Project Officer and three experts in
different areas of robotics. It received very positive critics. Part of the
presentation for the evaluation meeting consisted in a demonstration that
displayed basic prototypes of the different subsystems interacting with each
other, and was
able to solve very simple disassembly simulated scenarios
involving a
hard drive (see Fig.~\ref{fig:vrepimage}).
This demonstration already featured a planner prototype in the loop.

\section{Scope and goals}

The present work revolves mainly around probabilistic planning, without focusing too much
on topics such as scene perception and low-level robot control.
We present tools for modeling stochastic problems from the recycling
domain, as well as techniques for
dealing efficiently with these problems. Although all the examples and benchmarks
have been conceived for the particular use case of the hard drive,
the techniques described here can be extrapolated to other devices by extending
the state specification with variables particular to these devices.
We focus on the use case of the
hard drive disassembly.

More specifically we:
\begin{itemize}
	\item propose a set of predicates to represent the different components of
	the device and their relation to other components. These are used to represent
	the world state.
	\item propose a PPDDL (\emph{Probabilistic Planning Domain Definition Language})
	model for the recycling domain. This model contains the schemata of the actions
	that can be applied in the domain, as well as the specification of the
	state variables (predicates and functions). PPDDL defines an action-based
	DBN (\emph{Dynamic Bayes Network}) with full observability or, equivalently,
	a MDP.
	\item explore and test several determinization techniques to achieve fast
	approximate on-line solutions to MDPs. The rationale is to avoid costly policy
	recalculations each time the state specification changes when new objects
	are discovered (e.g. after removing the lid of a hard drive). The considered
	determinization algorithms are: (1) All-Outcome and Single-Outcome;
	(2) Alpha-Cost Transition Likelihood; (3) Hindsight optimization.
	\item complement the previous technique with hierarchical planning, taking
	advantage of the strong precedences that arise naturally in the
	considered application. This occurs, for instance, when the read
\end{itemize}

We assess the effectiveness of these strategies testing them against
benchmark problems. On the one hand, we have all the problems from past
IPPCs
(\emph{International Probabilistic Planning Competitions}, hosted
each year by the ICAPS). These are useful to evaluate the suitability of
the different
determinization strategies in domains different from ours. On the other hand,
we have hand-crafted a set of benchmark problems that consist of
hard drive variations with different disposition of components. The determinization
algorithms are evaluated on top of these, too. In addition, these will serve
the purpose of assessing the computational gain of the hierarchical planning
method that is specific for the recycling domain.

\section{Related work}

asdfsadf

\IfEq{\jobname}{\detokenize{root}}{}{\printbibliography}

\end{document}