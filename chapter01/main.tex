\documentclass[../root.tex]{subfiles}

\begin{document}

\section{Introduction}

Industrial robots are programmed to perform highly specialized and
repetitive actions in controlled environments. Therefore, their
autonomy is quite limited and they are restricted to a small set
of problems with small disturbances. We would like that robots that are
not in such controlled scenarios are able solve a larger variety of
problems, and even to react in the presence
of adverse events.
In other words, we
would like robots to reason about its environment and about the
potential effects that their actions may exert on it.
Such capabilities would allow for great flexibility in the following
ways: (1) tasks can be switched without reprogramming; (2) selection
of actions that minimize the risk of adverse outcomes; and (3)
contingency mechanisms, should these harmful effects happen anyway.

Our work belongs to the area of \emph{Automatic Planning and Scheduling}
or, simply, planning. It is a branch inside Artificial
Intelligence that aims at solving problems that are defined
declaratively, or with very limited procedural knowledge. That is, a
planning algorithm
operates with a model that encodes the dynamics of the domain and, for
each problem, outputs
an action sequence
or a policy that solves it, has a high success probability or has a high
expected reward. Since Automatic Planning deals with optimizing an agent's
behavior in a certain environment, it is possible to see some
similarities between planning and Reinforcement Learning. The main difference
is that, in its most basic form of planning, the dynamics of the domain are fully
known by the agent and there is no learning involved (although definitively there
exist works that allow partial domains and that incorporate learning
capabilities to complete these domains~\cite{martinez2017relational, martinez2015vmin}).
There

We can think of several applications for robots that exhibit these
virtues: assistance of old or impaired people for household chores
or treatment~\cite{canal2018behavior,andriella2018deciding}; automatic system
configuration or repair in difficult-to-access environments like
subaquatic facilities~\cite{palomeras2016toward,ong2010planning} and; the one
in which we will focus: recycling (retrieval of the valuable components)

\section{Related projects}

\section{Problem formulation}

\section{Goals}

%\IfEq{\jobname}{\detokenize{root}}{}{\printbibliography}

\end{document}