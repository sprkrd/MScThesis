\documentclass[11pt,a4paper,oldfontcommands,oneside]{memoir}

\usepackage{tfm}
\usepackage[Glenn]{fncychap}
\usepackage[sorting=none]{biblatex}

\addbibresource{mscthesis.bib}

\begin{document}

%%%%%%%%%%%%%%
% FRONT PAGE %
%%%%%%%%%%%%%%

\begin{titlingpage}
	\TPGrid{100}{50}
	\AddToShipoutPicture*{\BackgroundPic}
	
	\begin{textblock*}{205mm}(0mm,6mm)
		{\bfseries\sffamily\color{white}\LARGE\null\hfill Institut de Rob\`otica i Inform\`atica Industrial}
	\end{textblock*}
	
	\begin{textblock*}{205mm}(0mm,162mm)
		{\scshape\sffamily\color{white}\Huge\noindent\null\hfill Towards effective planning strategies\\\\%
		\null\hfill for robots in recycling}
	\end{textblock*}
	
	\begin{textblock*}{205mm}(0mm,190mm)
		{\itshape\sffamily\color{white}\LARGE\noindent\null\hfill by Alejandro Su\'arez Hern\'andez \\\\%
		\null\hfill to obtain the M.Sc. in Artificial Intelligence}
	\end{textblock*}
	
	\begin{textblock*}{210mm}(0mm,220mm)
		\begin{center}
		\textbf{Co-directed by}\\
		Carme Torras Gen\'is \& Guillem Aleny\`a Ribas (IRI-CSIC-UPC)
		
		\vspace{0.5cm}
		
		\textbf{Supervised by}\\
		Cecilio Angulo Bah\'on (ESAII, UPC)
		
		\vspace{0.5cm}
		
		\textbf{To be defended on}\\
		Monday Juny 18, 2018
		
		\end{center}
	\end{textblock*}
	
	\mbox{} % dummy mbox so Latex cuts to the next page

\end{titlingpage}

\cleardoublepage

%%%%%%%%%%%%
% ABSTRACT %
%%%%%%%%%%%%

\cleardoublepage

\thispagestyle{plain}
\pagenumbering{gobble}
\begin{center}
	\Large
	\textbf{Towards Effective Planning Strategies for Robots in Recycling}
	
	\vspace{0.4cm}
	\large
	Thesis for the M.Sc. in Artificial Intelligence
	
	\vspace{0.4cm}
	\textbf{Alejandro Su\'arez Hern\'andez}
	
	\vspace{0.9cm}
	\textbf{Abstract}
\end{center}

In the present work we describe several ideas for planning under 
uncertainty. Our intention is to apply these ideas for recycling
electronic devices with a robotic arm. We formulate the problem as
a goal-based MDP (Markov Decision Process) or, equivalently, a SSPP
(Stochastic Shortest Path Problem). Typical algorithms to solve MDPs,
such as Value Iteration, suffer from the curse of
dimensionality when the state is composed of a large number of
variables. To address this, we study the suitability
of several determinization techniques. The deterministic version
of the problem is solved with state-of-the-art classical planners
such as Fast Forward or
Fast Downward. The advantage of such methods is that only a small
region of the state space is explored. Another observation that we
exploit is that the goals disassembly domain can be sorted quite
effortlessly thanks to the topological precedences between components.
Therefore, we can limit the planning horizon and plan hierarchically.

\vspace{0.33cm}
\noindent
{\itshape
	En este trabajo describimos varias ideas para planificar bajo
	incertidumbre. Queremos aplicar estas ideas a reciclar
	dispositivos electrónicos mediante un brazo robot. Formulamos
	el problema como un MDP (Markov Decision Process) o, equivalentemente,
	un SSPP (Stochastic Shortest Path Problem). Los algoritmos típicos
	de resolución de MDP, tales como Value Iteration, sufren de
	espacios de estados inmensos cuando el estado está conformado
	por una gran cantidad de variables. Para superar esta dificultad,
	estudiamos el redimiento de varias técnicas de determinización.
}

\vspace{1cm}
\noindent
{\itshape
	Descrivim un projecte d'investigaci\'o on explorem l'efectivitat d'una
	estat\`egia d'integraci\'o de planificaci\'o simb\'olica i geom\`etrica
	en l'àmbit de la rob\`otica. Ens proposem resoldre un problema equiparable
	a una tasca de muntatge. L'escenari que
	plantegem t\'e dos robots WAM de l'empresa Barrett Technology
	treballant cooperativament per resoldre un joc pels nens. L'experiment
	compleix dues missions: plantejar-nos un repte pr\`actic que ens ajudi a
	orientar la nostra feina; i proporcionar una forma visual de mostrar i
	validar els resultats obtinguts. A m\'es a m\'es presentem aspectes
	t\'ipics de la gesti\'o de projectes com per exemple la planificaci\'o
	temporal i l'an\`alisi social, econ\`omic i ambiental.
}
\cleardoublepage

% Reset page numbering to arabic

\pagenumbering{arabic}

%%%%%%%%%%%%%%%%%%%%%
% TABLE OF CONTENTS %
%%%%%%%%%%%%%%%%%%%%%

\tableofcontents

\cleardoublepage

%%%%%%%%%%%%
% CHAPTERS %
%%%%%%%%%%%%

%\chapter{auto chapter}
%
%\blindmathpaper

\subfile{chapter01/main}

\subfile{chapter02/main}

\printbibliography


\end{document}