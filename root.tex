\documentclass[11pt,a4paper,oldfontcommands,oneside]{memoir}

\usepackage{tfm}
\usepackage[Glenn]{fncychap}
\usepackage[sorting=none]{biblatex}

\addbibresource{mscthesis.bib}

\begin{document}

%%%%%%%%%%%%%%
% FRONT PAGE %
%%%%%%%%%%%%%%

\begin{titlingpage}
	\TPGrid{100}{50}
	\AddToShipoutPicture*{\BackgroundPic}
	
	\begin{textblock*}{205mm}(0mm,6mm)
		{\bfseries\sffamily\color{white}\LARGE\null\hfill Institut de Rob\`otica i Inform\`atica Industrial}
	\end{textblock*}
	
	\begin{textblock*}{205mm}(0mm,162mm)
		{\scshape\sffamily\color{white}\Huge\noindent\null\hfill Towards effective planning strategies\\\\%
		\null\hfill for robots in recycling}
	\end{textblock*}
	
	\begin{textblock*}{205mm}(0mm,190mm)
		{\itshape\sffamily\color{white}\LARGE\noindent\null\hfill by Alejandro Su\'arez Hern\'andez \\\\%
		\null\hfill to obtain the M.Sc. in Artificial Intelligence}
	\end{textblock*}
	
	\begin{textblock*}{210mm}(0mm,220mm)
		\begin{center}
		\textbf{Co-directed by}\\
		Carme Torras Gen\'is \& Guillem Aleny\`a Ribas (IRI-CSIC-UPC)
		
		\vspace{0.5cm}
		
		\textbf{Supervised by}\\
		Cecilio Angulo Bah\'on (ESAII, UPC)
		
		\vspace{0.5cm}
		
		\textbf{To be defended on}\\
		Monday Juny 18, 2018
		
		\end{center}
	\end{textblock*}
	
	\mbox{} % dummy mbox so Latex cuts to the next page

\end{titlingpage}

\cleardoublepage

%%%%%%%%%%%%
% ABSTRACT %
%%%%%%%%%%%%

\cleardoublepage

\thispagestyle{plain}
\pagenumbering{gobble}
\begin{center}
	\Large
	\textbf{Towards Effective Planning Strategies for Robots in Recycling}
	
	\vspace{0.4cm}
	\large
	Thesis for the M.Sc. in Artificial Intelligence
	
	\vspace{0.4cm}
	\textbf{Alejandro Su\'arez Hern\'andez}
	
	\vspace{0.9cm}
	\textbf{Abstract}
\end{center}

We describe a research project in which we explore the effectiveness of an
approach for integrated symbolic and geometric planning in robotics.
We target to solve an assembling-like problem with two robot arms. The
scenario we propose involves two \emph{Barrett Technology}'s WAM robots
that work cooperatively to solve a game for kids. This experiment has
a double purpose: setting out a practical challenge that guides our work; and
acting as a means to visually validate and show the obtained results.
We also cover Project Management aspects such as the temporal planning and
the economic, social and
environmental analysis.

\vspace{1cm}
\noindent
{\itshape
	Describimos un proyecto de investigaci\'on en el cual exploramos la
	efectividad de una estrategia de integraci\'on de planificaci\'on
	simb\'olica y geom\'etrica en el \'area de la rob\'otica. Nos proponemos
	resolver un problema equiparable a una tarea de ensamblado mediante dos
	brazos robot. El escenario que planteamos involucra dos robots WAM de la
	empresa Barrett Technology trabajando cooperativamente para resolver
	un juego dirigido a un p\'ublico infantil. El experimento cumple dos
	misiones: plantearnos un reto pr\'actico que nos ayude a orientar y guiar
	nuestro trabajo; y proporcionar un medio
	visual de demostrar y validar los resultados obtenidos. Adicionalmente
	cubrimos aspectos t\'ipicos de la gesti\'on de proyectos tales como
	la planificaci\'on temporal y el an\'alisis social, econ\'omico y
	medioambiental.
}

\vspace{1cm}
\noindent
{\itshape
	Descrivim un projecte d'investigaci\'o on explorem l'efectivitat d'una
	estat\`egia d'integraci\'o de planificaci\'o simb\'olica i geom\`etrica
	en l'àmbit de la rob\`otica. Ens proposem resoldre un problema equiparable
	a una tasca de muntatge. L'escenari que
	plantegem t\'e dos robots WAM de l'empresa Barrett Technology
	treballant cooperativament per resoldre un joc pels nens. L'experiment
	compleix dues missions: plantejar-nos un repte pr\`actic que ens ajudi a
	orientar la nostra feina; i proporcionar una forma visual de mostrar i
	validar els resultats obtinguts. A m\'es a m\'es presentem aspectes
	t\'ipics de la gesti\'o de projectes com per exemple la planificaci\'o
	temporal i l'an\`alisi social, econ\`omic i ambiental.
}
\cleardoublepage

% Reset page numbering to arabic

\pagenumbering{arabic}

%%%%%%%%%%%%%%%%%%%%%
% TABLE OF CONTENTS %
%%%%%%%%%%%%%%%%%%%%%

\tableofcontents

\cleardoublepage

%%%%%%%%%%%%
% CHAPTERS %
%%%%%%%%%%%%

%\chapter{auto chapter}
%
%\blindmathpaper

\subfile{chapter01/main}

\subfile{chapter02/main}

\printbibliography


\end{document}