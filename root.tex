\documentclass[11pt,a4paper,oldfontcommands,oneside]{memoir}

\usepackage{tfm}
\usepackage[Glenn]{fncychap}
\usepackage[sorting=none]{biblatex}

\addbibresource{mscthesis.bib}

\begin{document}

%%%%%%%%%%%%%%
% FRONT PAGE %
%%%%%%%%%%%%%%

\begin{titlingpage}
	\TPGrid{100}{50}
	\AddToShipoutPicture*{\BackgroundPic}
	
	\begin{textblock*}{205mm}(0mm,6mm)
		{\bfseries\sffamily\color{white}\LARGE\null\hfill Institut de Rob\`otica i Inform\`atica Industrial}
	\end{textblock*}
	
	\begin{textblock*}{205mm}(0mm,162mm)
		{\scshape\sffamily\color{white}\Huge\noindent\null\hfill Towards effective planning strategies\\\\%
		\null\hfill for robots in recycling}
	\end{textblock*}
	
	\begin{textblock*}{205mm}(0mm,190mm)
		{\itshape\sffamily\color{white}\LARGE\noindent\null\hfill by Alejandro Su\'arez Hern\'andez \\\\%
		\null\hfill to obtain the M.Sc. in Artificial Intelligence}
	\end{textblock*}
	
	\begin{textblock*}{210mm}(0mm,220mm)
		\begin{center}
		\textbf{Co-directed by}\\
		Carme Torras Gen\'is \& Guillem Aleny\`a Ribas (IRI-CSIC-UPC)
		
		\vspace{0.5cm}
		
		\textbf{Supervised by}\\
		Cecilio Angulo Bah\'on (ESAII, UPC)
		
		\vspace{0.5cm}
		
		\textbf{To be defended on}\\
		Monday Juny 18, 2018
		
		\end{center}
	\end{textblock*}
	
	\mbox{} % dummy mbox so Latex cuts to the next page

\end{titlingpage}

\cleardoublepage

%%%%%%%%%%%%
% ABSTRACT %
%%%%%%%%%%%%

\newcommand{\abstractseparation}{0.66cm}
\cleardoublepage

\thispagestyle{plain}
\pagenumbering{gobble}
\begin{center}
	\Large
	\textbf{Towards Effective Planning Strategies for Robots in Recycling}
	
	\vspace{0.4cm}
	\large
	Thesis for the M.Sc. in Artificial Intelligence
	
	\vspace{0.4cm}
	\textbf{Alejandro Su\'arez Hern\'andez}
	
	\vspace{0.9cm}
	\textbf{Abstract}
\end{center}

In this work we present several ideas for planning under 
uncertainty. Our intention is to apply these ideas to the domain
of recycling
electronic devices with a robotic arm. We formulate the problem as
a goal-based MDP (Markov Decision Process) or, equivalently, a SSPP
(Stochastic Shortest Path Problem). We propose determinization
techniques in order to take advantage of state-of-the-art
classical planning systems, such as Fast Downward. On the other hand,
we exploit the topological precedences between the different
components of the device to plan hierarchically and limit further
the complexity of (re)planning. This work has been conducted within
the European H2020 Imagine project.

\vspace{\abstractseparation}%
\hrule
\vspace{\abstractseparation}
\noindent
{\itshape
	En aquest treball presentem diverses idees per planificar sota
	incertesa. Volem aplicar aquestes idees a reciclar
	dispositius electr\`onics mitjan\c{c}ant un bra\c{c} robot. Formulem
	el problema com un MDP (Markov Decision Process) o, equivalentment,
	un SSPP (Stochastic Shortest Path Problem). Proposem t\`ecniques
	de determinitzaci\'o per aprofitar els \'ultims avan\c{c}os en
	planificadors cl\`assics com, per exemple, Fast Downward. D'altra banda,
	explotem les preced\`encies topol\`ogiques entre els diferents
	components del dispositiu per planificar jer\`arquicament i
	limitar encara m\'es la complexitat de (re)planificar. Aquest
	treball es s'ha dut a terme dins del projecte europeu H2020 Imagine.
}

\vspace{\abstractseparation}
\hrule
\vspace{\abstractseparation}
\noindent
{\itshape
	En este trabajo presentamos varias ideas para planificar bajo
	incertidumbre. Queremos aplicar estas ideas a reciclar
	dispositivos electr\'onicos mediante un brazo robot. Formulamos
	el problema como un MDP (Markov Decision Process) o, equivalentemente,
	un SSPP (Stochastic Shortest Path Problem). Proponemos t\'ecnicas
	de determinizaci\'on para aprovechar los últimos avances en
	planificadores cl\'asicos, tales como Fast Downward. Por otro lado,
	explotamos las precedencias topol\'ogicas entre los distintos
	componentes del dispositivo para planificar jer\'arquicamente y
	limitar a\'un m\'as la complejidad de (re)planificar. Este trabajo
	se ha llevado a dentro del marco del proyecto europeo H2020 Imagine.
}

\cleardoublepage

% Reset page numbering to arabic
\pagenumbering{arabic}

%%%%%%%%%%%%%%%%%%%%%
% TABLE OF CONTENTS %
%%%%%%%%%%%%%%%%%%%%%

\tableofcontents

\cleardoublepage

%%%%%%%%%%%%
% CHAPTERS %
%%%%%%%%%%%%

%\chapter{auto chapter}
%
%\blindmathpaper

\chapter{Scope and contextualization}
\subfile{chapter01/main}

\chapter{Theory and background}
\subfile{chapter02/main}

\chapter{Determinization of stochastic problems}
\subfile{chapter03/main}

\chapter{Short horizon planning}
\subfile{chapter04/main}

\chapter{Experimental evaluation}
\subfile{chapter05/main}

\chapter{Conclusions and future work}
\subfile{chapter06/main}

\printbibliography

\end{document}