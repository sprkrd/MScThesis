\documentclass[../root.tex]{subfiles}

\begin{document}

\section{Introduction}

Industrial robots are programmed to perform highly specialized and
repetitive actions in controlled environments. Therefore, their
autonomy is quite limited and they are restricted to a small set
of problems with small disturbances. We would like that robots that are
not in such controlled scenarios are able solve a large variety of
problems, and even to react in the presence
of adverse events.
In other words, we
would like robots to reason about its environment and about the
potential effects that their actions may exert on it.
Such capabilities would allow for great flexibility in the following
ways: (1) large number of problems that can be handled, where a problem
is determined by the disposition of the objects that the robot has to
interact with and by the goal condition; (2) anticipatory
action selection to avoid undesired effects whenever possible; and (3)
contingency mechanisms, should these harmful effects happen anyway.

We can think of several applications for robots that exhibit these
virtues: assistance of old or impaired people for household chores
or treatment; automatic system
configuration or repair in difficult-to-access environments like
subaquatic facilities; robot-assisted search and rescue missions
in which several locations has to be explored and debris has to be
moved around.

We tackle this \emph{Automatic Planning and Scheduling} is a branch inside Artificial
Intelligence that aims at solving problems that are defined
declaratively, or almost declaratively. That is, a planning algorithm
operates with a model of the problem and outputs an action sequence
or a policy that solves this 

\section{Related projects}

\section{Problem formulation}

\section{Goals}

\IfEq{\jobname}{\detokenize{root}}{}{\printbibliography}

\end{document}